\section{Introduction}

Emotion recognition is a rapidly growing field that plays a pivotal role in advancing human-computer interactions. By enabling machines to interpret and respond to human emotions, it opens up possibilities for more natural and empathetic interfaces across diverse applications, such as healthcare, customer service, education, and entertainment \cite{multimodal-review, unimodal-to-multimodal}. The challenge lies in accurately identifying emotional states, which are often expressed through a combination of verbal and non-verbal cues, including spoken words, vocal tone, facial expressions, and body language \cite{multimodal-review, basic-emotions}.

While traditional emotion recognition systems typically rely on unimodal data, such as text or audio alone, they often fail to capture the complexity and nuances of human emotions \cite{basic-emotions, unimodal-to-multimodal}. Multi-modal emotion recognition systems have emerged as a promising alternative, integrating multiple data sources to enhance accuracy and robustness \cite{multimodal-review, basic-emotions}. By combining information from text, audio, and visual modalities, these systems leverage the complementary nature of various emotional expressions to provide a more holistic understanding of emotional states \cite{multimodal-review, unimodal-to-multimodal}.

Despite advances in multi-modal recognition, several challenges persist. One major obstacle is the effective fusion of data from different modalities while preserving their individual characteristics and capturing their inter-modal relationships \cite{multimodal-review, basic-emotions}. Additionally, many datasets used to train these systems are curated in controlled environments, potentially limiting their generalizability to real-world interactions \cite{multimodal-review, unimodal-to-multimodal}.. Addressing these challenges requires innovative models and methodologies that are both computationally efficient and capable of extracting meaningful features from various data sources \cite{basic-emotions}.

This project explores the development of a multi-modal emotion recognition system designed to overcome these limitations \cite{multimodal-review, unimodal-to-multimodal}.. Through the integration of advanced feature extraction techniques and hybrid fusion strategies, it aims to improve the accuracy and generalizability of emotion recognition models \cite{multimodal-review, basic-emotions}. By building on state-of-the-art research, this work seeks to contribute to the ongoing evolution of more intelligent and emotionally aware systems \cite{basic-emotions, unimodal-to-multimodal}.

 

 